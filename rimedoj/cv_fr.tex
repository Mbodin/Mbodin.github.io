\documentclass[12pt,a4paper]{article}

\usepackage{ifxetex}
\ifxetex
\usepackage{fontspec}
\usepackage{polyglossia}
\setmainlanguage{french}
\setotherlanguage{english}
\setotherlanguage{spanish}
\newcommand{\en}[1]{\foreignlanguage{english}{{#1}}}
\newcommand{\es}[1]{\foreignlanguage{spanish}{{#1}}}
\else
\usepackage[utf8x]{inputenc}
\usepackage[T1]{fontenc}
\usepackage[francais, english, spanish]{babel}
\newcommand{\en}[1]{\foreignlanguage{english}{{#1}}}
\newcommand{\es}[1]{\foreignlanguage{spanish}{{#1}}}
\fi

\usepackage{geometry}
\usepackage{url}
\usepackage{oldgerm}

\usepackage[inline]{enumitem}

\makeatletter
% From http://tex.stackexchange.com/questions/51086/add-second-enumeration-item-on-the-same-line
\newcommand{\inlineitem}[1][]{%
\ifnum\enit@type=\tw@
    {\descriptionlabel{#1}}
  \hspace{\labelsep}%
\else
  \ifnum\enit@type=\z@
       \refstepcounter{\@listctr}\fi
    \qquad\@itemlabel\hspace{\labelsep}%
\fi}
\makeatother

\ifxetex
\setmainfont[Mapping=tex-text, Ligatures={Common}, Numbers={OldStyle}]{Linux Libertine O}
\fi

\geometry{
	includeheadfoot,
	margin = 2.54cm,
	top = 1.5cm,
	bottom = 1.5cm
}

\newcommand{\ie}{i.e.\,}

\newenvironment{datecvsection}[1]%
               {\subsection*{#1}%
                 \noindent \begin{tabular}{@{}p{\annee}p{\texte}@{}}}
               {\end{tabular}}

\newenvironment{cvsection}[1]%
               {\subsection*{#1}}
               {}

\newenvironment{itemcvsection}[1]%
               {\subsection*{#1}\begin{itemize}}
               {\end{itemize}}

\def\arraystretch{1.5}
 
\newcommand\familyName{\textsc}
\newcommand\placeName{}

\begin{document}

% Pas de numéro de page
\pagestyle{empty}

% \annee est la largeur de la première colonne, c'est à dire celle
% contenant l'année scolaire. Elle est ici définie comme étant la
% largeur du texte « janvier-février ». À adapter le cas échéant.

\newlength{\annee}
\settowidth{\annee}{9999--9999}

% \texte est la largeur de la deuxième colonne. Elle est définie comme
% étant la largeur de la page moins celle de la première colonne.
% 2\tabcolsep est la largeur de l'espacement entre les colonnes.

\newlength{\texte}
\setlength{\texte}{\textwidth} \addtolength{\texte}{-\annee} 
	\addtolength{\texte}{-2\tabcolsep}

\begin{center} \Huge Martin~\familyName{Bodin} \end{center}

\parbox[c]{.5\textwidth}
{
  \noindent
  Appartement 243, \\
  55/57 Sloane Avenue, \\
  SW3~3AZ, London, \\
  \textsc{Royaume-Uni}
}
\parbox[c]{.55\textwidth}
{
\begin{flushright}
  Né le 30 Décembre 1989 \\
  \noindent Téléphone~: \mbox{+33 6 27 00 58 84} \\
  \url{martin.bodin@ens-lyon.org} \\
  \url{https://www.doc.ic.ac.uk/~mbodin}
\end{flushright}
}


\begin{datecvsection}{Positions}

    2018–2020 & Postdoc à l’\en{\placeName{Imperial College}}, à \placeName{Londres} (\placeName{Royaume-Uni}) sur les fondations théoriques de l’analyse de langages de programmation complexes. \\

    2017–2018 & Postdoc au \placeName{CMM} (le \en{\placeName{Center of Mathematical Modeling}}, attaché à l’\es{\placeName{Universidad de Chile}}), à \placeName{Santiago} (\placeName{Chili}), sur la formalisation du langage de programmation R dans l’assistant de preuve Coq. \\

\end{datecvsection}

\begin{datecvsection}{Études}

    2012–2016 & Doctorat en Informatique à l’\placeName{IRISA}, à \placeName{Rennes}, sous la direction d’Alan~\familyName{Schmitt} et Thomas~\familyName{Jensen} sur les analyses certifiées de JavaScript. \\

	2010–2012 & Master d’Informatique, mention bien à l’École Normale Supérieure de \placeName{Lyon}. \\

	2009–2010 & Licence d’Informatique, mention très bien à l’École Normale Supérieure de \placeName{Lyon}. \\

    2007–2009 & Classes préparatoires aux Grandes Écoles au lycée \placeName{Henri~\textsc{iv}} à \placeName{Paris}. \\

	2007 & Baccalauréat spécialité Mathématiques, mention très bien. \\

\end{datecvsection}

\begin{itemcvsection}{Langues}

   \item Français langue maternelle,
   \item Bon niveau d’anglais,
   \item Très bon niveau d’espéranto,
   \item Niveau moyen de portugais,
   \item Bas niveau d’espagnol,
   \item Notions d’allemand.

\end{itemcvsection}

\begin{datecvsection}{Conférences}

  2019 & Martin~\familyName{Bodin}, Philippa~\familyName{Gardner}, Thomas~\familyName{Jensen}, et Alan~\familyName{Schmitt}. \textit{Skeletal Semantics et Their Interpretations}. À POPL. \\

  2018 & Martin~\familyName{Bodin}, Tomás~\familyName{Diaz}, et Éric~\familyName{Tanter}. \textit{A Trustworthy Mechanized Formalization of R}. À DLS. \\

  2015 & Martin~\familyName{Bodin}, Thomas~\familyName{Jensen}, et Alan~\familyName{Schmitt}. \textit{Certified Abstract Interpretation with Pretty-Big-Step Semantics}. À CPP. \\

  2014 & Martin~\familyName{Bodin}, Arthur~\familyName{Charguéraud}, Daniele~\familyName{Filaretti}, Philippa~\familyName{Gardner}, Sergio~\familyName{Maffeis}, Daiva~\familyName{Naudžiūnienė}, Alan~\familyName{Schmitt}, et Gareth~\familyName{Smith}. \textit{A Trusted Mechanised JavaScript Specification}. À POPL. \\

  2013 & Jörg~\familyName{Endrullis}, Dimitri~\familyName{Hendriks}, et Martin~\familyName{Bodin}. \textit{Circular Coinduction in Coq Using Bisimulation-up-to Techniques}. À ITP. \\

  2011 & David~\familyName{Monniaux} et Martin~\familyName{Bodin}. \textit{Modular Abstractions of Reactive Nodes using Disjunctive Invariants}. À APLAS. \\

\end{datecvsection}

\begin{datecvsection}{Colloques}

  2019 & Martin~\familyName{Bodin}, Tomás~\familyName{Diaz}, et Éric~\familyName{Tanter}. \textit{A Trustworthy Mechanized Formalization of R}. À FMfSS. \\

  2018 & Martin~\familyName{Bodin}. \textit{A Coq Formalisation of a Core of R}. À CoqPL. \\

  2016 & Martin~\familyName{Bodin}, Thomas~\familyName{Jensen}, et Alan~\familyName{Schmitt}. \textit{An Abstract Separation Logic for Interlinked Extensible Records}. À JFLA. \\

  2014 & Martin~\familyName{Bodin}, Thomas~\familyName{Jensen}, et Alan~\familyName{Schmitt}. \textit{Certified Abstract Interpretation with Pretty-Big-Step Semantics}. À JFLA. \\

  2013 & Martin~\familyName{Bodin}, Thomas~\familyName{Jensen}, et Alan~\familyName{Schmitt}. \textit{Pretty-big-step-semantics-based Certified Abstract Interpretation (Preliminary version)}. À \textit{Semantics, Abstract Interpretation, et Reasoning about Programs: Essays Dedicated to David A. \textsc{Schmidt}} \\ % on the Occasion of his Sixtieth Birthday}, volume 129 of EPTCS, pages 360–383. \\

  2013 & Martin~\familyName{Bodin} et Alan~\familyName{Schmitt}. \textit{A Certified JavaScript Interpreter}. À JFLA. \\

\end{datecvsection}

\begin{datecvsection}{Autres Publications}

    2016 & Martin~\familyName{Bodin}. \textit{Sémantique et analyse certifiées de JavaScript}. Thèse. \\

\end{datecvsection}

\begin{datecvsection}{Prises de responsabilité}

    2019 & Membre du comité d’évaluation des artéfacts pour POPL. \\

    2012, 2014 & Surveillance de l’épreuve pratique d’algorithmique et de programmation du concours d’entrée à l’\placeName{École Normale Supérieure}.

\end{datecvsection}

\begin{datecvsection}{Encadrement}

    2020 & Rao~\familyName{Xiaojia} et Vincent~\familyName{Rébiscoul}, co-encadrés avec Philippa~\familyName{Gardner} \\
    2019 & Benjamin~\familyName{Gunton}, co-encadré avec Philippa~\familyName{Gardner} \\
    2017–2018 & Tomás~\familyName{Diaz}, co-encadré avec Éric~\familyName{Tanter} \\

\end{datecvsection}

\begin{datecvsection}{Enseignement}

    2019 & Separation Logic. Travaux dirigés de M1 à l’\en{\placeName{Imperial College}}. \\

    2018 & Models of Computation. Travaux dirigés de L2 à l’\en{\placeName{Imperial College}}. \\

    2017 & Introducción a Coq: Lógica, Tipos y Verificación. Travaux pratiques de M2 à l’\es{\placeName{Universidad de Chile}}. \\

    2014 & Conception et vérification formelle de programmes. Travaux dirigés de M1 à l’\placeName{École Normale Supérieure}. \\

    2013–2014 & Langages formels. Travaux dirigés de L3 à l’\placeName{École Normale Supérieure}. \\

    2013–2014 & Introduction à la programmation fonctionnelle. Travaux pratiques de L1 à l’université \placeName{Rennes}~1. \\

    2012 & Introduction à Scheme. Travaux pratiques et dirigés de L1 à l’\placeName{Insa} de \placeName{Rennes}. \\

\end{datecvsection}

\begin{datecvsection}{Autres expériences}

	2013–2016 & Membre du projet \textsc{SecCloud}~:  \url{http://www.seccloud.cominlabs.ueb.eu/} \\

	2012–2016 & Membre du projet \textsc{JsCert}~:  \url{http://jscert.org/} \\

    2012 & Stage à \placeName{Inria} (\placeName{Rennes}) avec Alan~\familyName{Schmitt} et Thomas~\familyName{Jensen}~:
	\en{\textit{A Certified JavaScript Interpreter}} \\

    2011 & Stage à \placeName{Vrije Universiteit} (\placeName{Amsterdam}) avec Dimitri~\familyName{Hendriks} et Jörg~\familyName{Endrullis}~:
	\en{\textit{Proving Stream Equalities in Coq}} \\

    2011 & Stage à \placeName{PPS} (\placeName{Paris}) avec Stéphane~\familyName{Gimenez} et Christine~\familyName{Tasson}~:
	\en{\textit{Sequentialization of Proof Nets to Structures}} \\

	2010 & Participation au projet \textsc{Cartomancer}~:  \url{http://sourceforge.net/projects/cartomancer/} \\

    2010 & Stage à \placeName{Vérimag} (\placeName{Grenoble}) avec David~\familyName{Monniaux}~:
	\textit{Détection de modes de fonctionnements d’un programme \textsc{Lustre}} \\

\end{datecvsection}

\begin{itemcvsection}{Compétences informatiques}

  \item Bon niveau dans les langages de programmation suivants~:  Coq, JavaScript, R, \LaTeX, OCaml, etc.
  \item Familier avec le système d’exploitation GNU/Linux.
  \item Capable d’utiliser Vim, Git, The~Gimp, Inkscape, etc.

\end{itemcvsection}

\begin{datecvsection}{Vie associative}

    2017 & Rencontres espérantistes à \placeName{Santiago}. \\
    2014–2015 & Membre actif d’{Espéranto-Rennes}. \\
	2012 & Trésorier de l’association de théâtre l’{ASCREB} à \placeName{Rennes}. \\
    2010 & Secrétaire de l’association \textsc{INFO-ENS} à \placeName{Lyon}. \\

\end{datecvsection}

\begin{itemcvsection}{Intérêts personnels}

  \item Jeux de plateaux, de cartes, de rôles, d’improvisation, soirées enquêtes, etc.
  \item Activités espérantistes.

\end{itemcvsection}

\end{document}

