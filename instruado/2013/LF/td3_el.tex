\documentclass[11pt,a4paper]{article}
  \usepackage[T1]{fontenc}%
  \usepackage[obeyspaces]{url} % \urlstyle{sf}%
  \usepackage{mathpazo}%
\usepackage[dvips]{epsfig}
\usepackage[latin1]{inputenc}

\usepackage{fancyhdr}
\usepackage[latin1]{inputenc}
\usepackage{graphicx,multicol,amsmath,amssymb,ifthen}
\usepackage[francais]{babel} \AutoSpaceBeforeFDP %
\usepackage{color}
\usepackage{framed}
\usepackage{stmaryrd}


\setlength{\oddsidemargin}{0cm}
\addtolength{\textwidth}{70pt}
\setlength{\topmargin}{0cm}
\addtolength{\textheight}{3cm}


\newcommand{\Correction}[1]
{
  \ifthenelse{\boolean{corrige}}
  {\definecolor{shadecolor}{gray}{0.80} \begin{shaded} {\small #1} \end{shaded}}{}
}



\newcommand{\titreTD}[3]{%
\title{Module #1 \\
#2 : #3
\Correction{Correction}}
\author{}
\date{}

\pagestyle{fancy}
\lhead{\sf MIT 1}
\rhead{\sf Module #1 \ #2 \ifthenelse{\boolean{corrige}}{, correction}{}}}



\newcounter{probleme} 
\newcounter{question}
\setcounter{probleme}{1} 
\setcounter{question}{1} 
\def\theprobleme{\Roman{probleme}}
\def\thequestion{\arabic{question}}

\newenvironment{probleme}[1]%
{\setcounter{question}{1}
\stepcounter{probleme}
\subsection*{\theprobleme. #1}}%
{}


\newcounter{excounter}
\newcounter{questcounter}[excounter]
\newenvironment{exer}[1][]%
{%
  \setcounter{question}{1}%
  \stepcounter{excounter}%
  \vspace*{0.6cm}\noindent { \texttt{{\bf Exercice \arabic{excounter}} #1}}\\
}{%
  \vspace{.3cm}%
}



\newenvironment{question}[1][]{\setcounter{numsubquestion}{1}%
\vspace*{0.5cm}\noindent{\bf {\large\theexcounter.\thequestion}. #1}}%
{\stepcounter{question}\medskip}

\newcounter{numdef}
\setcounter{numdef}{0}




\newenvironment{Def}[2]{%
\definecolor{shadecolor}{gray}{0.90}
\begin{shaded}\refstepcounter{numdef}%\vspace*{0.1cm}
    \noindent{ {\label{#2}\bf D�finition}~: {\it #1}} \\}%
{\end{shaded}\medskip}


\newenvironment{Th}[1]{%
\begin{framed}\vspace*{0.1cm}\noindent{ {\bf Th�or�me (#1)} :} \\}%
{\end{framed}}

\newenvironment{Lemma}[1]{%
\begin{framed}%\vspace*{0.1cm}
    \noindent{ {\bf Lemme (#1)} :} \\}%
{\end{framed}}


\newenvironment{Prop}[0]{%
\begin{framed}\vspace*{0.1cm}\noindent{ {\bf Proposition} :} \\}%
{\end{framed}}



\newcounter{numsubquestion}
\setcounter{numsubquestion}{1}
\newenvironment{subquestion}{\setcounter{numsubsubquestion}{1}%
\vspace*{0.2cm}\noindent{ {\bf\theexcounter.\thequestion . \thenumsubquestion}}.}%
{\stepcounter{numsubquestion}\medskip}


\newcounter{numsubsubquestion}
\newenvironment{subsubquestion}{\vspace*{0.1cm}\noindent{ {\bf\theexcounter.\thequestion.\thenumsubquestion.\thenumsubsubquestion}}.}%
{\stepcounter{numsubsubquestion}\medskip}


\let\bfseriesaux=\bfseries%
\renewcommand{\bfseries}{\sffamily \bfseriesaux}

\def\|{\leavevmode\url|}



\newenvironment{remarque}%
{\subparagraph{Remarque.}}%
{}

\newenvironment{programme}%
{\verbatim}{\endverbatim}

\newboolean{corrige} 

\setlength{\parindent}{0cm}

\newsavebox{\majbox}
\newlength{\majhaut}
\newenvironment{solution}{\color[gray]{0.5}%
  \begin{lrbox}{\majbox}
    \begin{minipage}[b]{.9\textwidth}}{%
   \end{minipage}
  \end{lrbox}
  \settoheight{\majhaut}{\usebox{\majbox}}
   \addtolength{\majhaut}{0.3mm}
  \ifthenelse{\boolean{corrige}}{%
    \rule{1mm}{\majhaut}\hspace*{2mm}\usebox{\majbox}}{}}

\newcommand{\st}{\ |\ }

\renewcommand\geq\geqslant
\renewcommand\leq\leqslant

 
  
\titreTD{Langages Formels}{TD 3}{Minimisation et langages rationnels}

\setboolean{corrige}{true}

% \documentclass[11pt]{article}

 

\parskip=0.3\baselineskip \sloppy


\newcommand{\Rat}{\mathsf{Rat}}%
\newcommand{\N}{\mathbb{N}}%
\newcommand{\D}{\mathbb{D}}%
\newcommand{\dist}{\mathop{\mathrm{d}}}%
\newcommand{\fix}{\mathop{\mathrm{Fix}}}%

\begin{document}

\thispagestyle{empty}
\maketitle




\begin{exer}[Minimisons !]
  \begin{question}
    Minimiser les automates suivants en utilisant l'algorithme vu en cours :
    \begin{center} \begin{tabular}{cc}
	\input{auto1.pstex_t} &
	\input{auto2.pstex_t} 
      \end{tabular}
    \end{center}

\Correction{

    % � compl�ter.

}

  \end{question}

  \begin{question}
    D�terminiser et minimiser l'automate suivant.
    � votre avis si on g�n�ralise � $n$ �tats, combien
    d'�tats aura le d�terminis� ? Le minimal ?
    \begin{center} \input{automax.pstex_t} \end{center}  

\Correction{

}

  \end{question}
\end{exer}



\begin{exer}[Trois Lemmes pour les �tudiants sous le ciel]

  Dans cet exercice, nous allons consid�rer les trois versions du lemme de l'�toile :
  \begin{enumerate}
  \item Si $L$ est un langage reconnu par un automate fini, alors 
    $$\exists n \in \N, \ \ \forall u \in L,\ |u| \ge n \ \Longrightarrow \ \exists v,t,w \in \Sigma^*, \quad u = vtw \ \land \ \
    |t| >0 \ \land \ \forall m \in \N, \ vt^mw \in L$$
  \item Si $L$ est un langage reconnu par un automate fini, alors 
    $$\exists n \in \N, \ \ \forall u=rst \in L ,\ |s| \ge n \ \Longrightarrow \ \exists v,w,x \in \Sigma^*, \quad s= vxw \land \ \
    |x| >0 \ \land \ \forall m \in \N, \ rvx^mwt \in L$$
  \item Si $L$ est un langage reconnu par un automate fini, alors 
    $$
\begin{array}{r}
\exists n \in \N, \ \ \forall u=ru_1 u_2 \dots u_n s \in L, \ (\forall i, |u_i| \ge 1) \ \Longrightarrow \ \exists 1 \le i
    < j \le n, \quad \forall m \in \N, \\
 ru_1\dots u_{i-1}(u_i\dots u_j)^m u_{j+1}\dots u_n s \in L
\end{array}
$$
  \end{enumerate}

  \begin{question}
    Soit $L = \{ u \in \{a,b\}^* \st |u|_a = |u|_b \}$, montrer que $L$ v�rifie le
    Lemme~1 mais pas le Lemme~2.
  \end{question}



  \Correction{

} 


  \begin{question}
    Soit $L' = \{(ab)^n (cd)^n \st n \in \N \} \cup \Sigma^* (aa+bb+cc+dd+ac+bd) \Sigma^*$, avec $\Sigma = \{a,b,c,d
    \}$. Montrer que $L'$ v�rifie le Lemme~2 mais pas le Lemme~3.
  \end{question}

  \Correction{

    }



\end{exer}



\begin{exer}[Ceci n'est pas un lemme de l'�toile]

\begin{question}
  Soit $L$ un langage reconnaissable. Montrer qu'il existe $N > 0$ tel que

\begin{equation*}
\begin{gathered}
\forall u \in \Sigma^*, |u| \ge N, \exists x,y,z \in \Sigma^*, |y| \ge 1, u = xyz \land \\
\forall i \ge 0, \ \forall v \in \Sigma^*, \ (uv \in L \iff xy^izv \in L)
\end{gathered}
\end{equation*}


\Correction{

}

\end{question}


\begin{question}
Soit un langage $L \subseteq \Sigma^*$ tel qu'il existe $N>0$ tel que
\begin{equation*}
\begin{gathered}
\forall u \in \Sigma^*, |u| \ge N, \exists x,y,z \in \Sigma^*, |y| \ge 1, u = xyz \land \\
\forall i \ge 0, \ \forall v \in \Sigma^*, \ (uv \in L \iff xy^izv \in L)
\end{gathered}
\end{equation*}

Montrer que $L$ est reconnaissable.

%%% Indication :
%% On pourra construire un automate
%% $A = (Q, \Sigma, \delta, q_0, F)$ tel que
%% $Q = \{q_w \st |w| < N\}$ qui reconna�t $L$.


\Correction{

}



\end{question}


\end{exer}



\begin{exer}[Le barman aveugle]


  On dispose de 4 jetons, chacun ayant une face bleue et une face rouge. Un joueur (le barman) a les
  yeux band�s. Son but est de retourner les 4 jetons sur la m�me couleur.
  D�s que les 4 jetons sont retourn�s, la partie s'arr�te et le barman a gagn�.

  Pour cela, il peut retourner � chaque tour 1, 2 ou 3 jetons.
  Un autre joueur perturbe le jeu en tournant le plateau sur lequel reposent les jetons
  d'un quart de tour, d'un demi-tour ou de trois quarts de tour entre chaque op�ration du barman.

  \begin{center}
    \input{barman.pstex_t}
  \end{center}

  En utilisant une mod�lisation par des automates, montrer que le barman a une strat�gie gagnante,
  c'est-�-dire que quoi que fasse le tourneur de plateau, il a moyen de gagner.

  \Correction{


  }
  



\end{exer}




\end{document}


