\documentclass[11pt,a4paper]{article}
  \usepackage[T1]{fontenc}%
  \usepackage[obeyspaces]{url} % \urlstyle{sf}%
  \usepackage{mathpazo}%
\usepackage[dvips]{epsfig}
\usepackage[latin1]{inputenc}

\usepackage{fancyhdr}
\usepackage[latin1]{inputenc}
\usepackage{graphicx,multicol,amsmath,amssymb,ifthen}
\usepackage[francais]{babel} \AutoSpaceBeforeFDP %
\usepackage{color}
\usepackage{framed}
\usepackage{stmaryrd}


\setlength{\oddsidemargin}{0cm}
\addtolength{\textwidth}{70pt}
\setlength{\topmargin}{0cm}
\addtolength{\textheight}{3cm}


\newcommand{\Correction}[1]
{
  \ifthenelse{\boolean{corrige}}
  {\definecolor{shadecolor}{gray}{0.80} \begin{shaded} {\small #1} \end{shaded}}{}
}



\newcommand{\titreTD}[3]{%
\title{Module #1 \\
#2 : #3
\Correction{Correction}}
\author{}
\date{}

\pagestyle{fancy}
\lhead{\sf MIT 1}
\rhead{\sf Module #1 \ #2 \ifthenelse{\boolean{corrige}}{, correction}{}}}



\newcounter{probleme} 
\newcounter{question}
\setcounter{probleme}{1} 
\setcounter{question}{1} 
\def\theprobleme{\Roman{probleme}}
\def\thequestion{\arabic{question}}

\newenvironment{probleme}[1]%
{\setcounter{question}{1}
\stepcounter{probleme}
\subsection*{\theprobleme. #1}}%
{}


\newcounter{excounter}
\newcounter{questcounter}[excounter]
\newenvironment{exer}[1][]%
{%
  \setcounter{question}{1}%
  \stepcounter{excounter}%
  \vspace*{0.6cm}\noindent { \texttt{{\bf Exercice \arabic{excounter}} #1}}\\
}{%
  \vspace{.3cm}%
}



\newenvironment{question}[1][]{\setcounter{numsubquestion}{1}%
\vspace*{0.5cm}\noindent{\bf {\large\theexcounter.\thequestion}. #1}}%
{\stepcounter{question}\medskip}

\newcounter{numdef}
\setcounter{numdef}{0}




\newenvironment{Def}[2]{%
\definecolor{shadecolor}{gray}{0.90}
\begin{shaded}\refstepcounter{numdef}%\vspace*{0.1cm}
    \noindent{ {\label{#2}\bf D�finition}~: {\it #1}} \\}%
{\end{shaded}\medskip}


\newenvironment{Th}[1]{%
\begin{framed}\vspace*{0.1cm}\noindent{ {\bf Th�or�me (#1)} :} \\}%
{\end{framed}}

\newenvironment{Lemma}[1]{%
\begin{framed}%\vspace*{0.1cm}
    \noindent{ {\bf Lemme (#1)} :} \\}%
{\end{framed}}


\newenvironment{Prop}[0]{%
\begin{framed}\vspace*{0.1cm}\noindent{ {\bf Proposition} :} \\}%
{\end{framed}}



\newcounter{numsubquestion}
\setcounter{numsubquestion}{1}
\newenvironment{subquestion}{\setcounter{numsubsubquestion}{1}%
\vspace*{0.2cm}\noindent{ {\bf\theexcounter.\thequestion . \thenumsubquestion}}.}%
{\stepcounter{numsubquestion}\medskip}


\newcounter{numsubsubquestion}
\newenvironment{subsubquestion}{\vspace*{0.1cm}\noindent{ {\bf\theexcounter.\thequestion.\thenumsubquestion.\thenumsubsubquestion}}.}%
{\stepcounter{numsubsubquestion}\medskip}


\let\bfseriesaux=\bfseries%
\renewcommand{\bfseries}{\sffamily \bfseriesaux}

\def\|{\leavevmode\url|}



\newenvironment{remarque}%
{\subparagraph{Remarque.}}%
{}

\newenvironment{programme}%
{\verbatim}{\endverbatim}

\newboolean{corrige} 

\setlength{\parindent}{0cm}

\newsavebox{\majbox}
\newlength{\majhaut}
\newenvironment{solution}{\color[gray]{0.5}%
  \begin{lrbox}{\majbox}
    \begin{minipage}[b]{.9\textwidth}}{%
   \end{minipage}
  \end{lrbox}
  \settoheight{\majhaut}{\usebox{\majbox}}
   \addtolength{\majhaut}{0.3mm}
  \ifthenelse{\boolean{corrige}}{%
    \rule{1mm}{\majhaut}\hspace*{2mm}\usebox{\majbox}}{}}

\newcommand{\st}{\ |\ }

\renewcommand\geq\geqslant
\renewcommand\leq\leqslant

 
  
\titreTD{Langages Formels}{TD 1}{Mots et langages} 

\setboolean{corrige}{true}


\parskip=0.3\baselineskip \sloppy


\newcommand{\N}{\mathbb{N}}%
\newcommand{\D}{\mathbb{D}}%
\newcommand{\dist}{\mathop{\mathrm{d}}}%
\newcommand{\fix}{\mathop{\mathrm{Fix}}}%
\newcommand\abs[1]{\left\lvert #1 \right\rvert}

\begin{document}

\thispagestyle{empty}
\maketitle



\begin{exer}[Une histoire de moutons\ldots]

    Soit $\Sigma$ un alphabet non vide.
    Montrez que le language $\Sigma^*$ est infini d�nombrable.

\end{exer}

\begin{exer}[R�visons les conjugaisons]


Deux mots $u$ et $v$ sur un alphabet $\Sigma$ sont dits {\bf
  conjugu�s} s'il existe des mots $s$ et $t$ sur $\Sigma$ tels que
$u = st$ et $v = ts$. 


\begin{question}
Montrer que la relation binaire $\sim$ sur $\Sigma^*$ d�finie par $u
\sim v $ ssi $u$ et $v$ sont conjugu�s est une relation d'�quivalence.
\end{question}


\begin{question}
Montrer que pour tout $n \geq 1$, $u \sim v \iff u^n \sim v^n$.
\end{question}

\begin{question}
Montrer que $u \sim v$ si et seulement s'il existe un mot $w$ tel que
$uw=wv$.
\end{question}

\end{exer}


\begin{exer}[On and On and On]


  On appelle {\bf code} sur un alphabet $\Sigma$ tout langage $X$ sur
  $\Sigma$ tel que pour toutes familles $(x_i) \in X^{\llbracket 1, p \rrbracket}$
  et $(y_i) \in X^{\llbracket 1, q \rrbracket}$,
  $x_1x_2 \ldots x_p = y_1y_2 \ldots y_q$ entraine $p = q$ et
  $x_i = y_i$ pour tout $i$. Dire que $X$ est un code revient donc �
  dire que tout \'el\'ement de $X^*$ se factorise de mani�re unique sur
  $X$.
  
  \begin{question}
    Les langages suivants sont-ils des codes ?
    \begin{itemize}
    \item $X_1 = \{ab, baa, abba, aabaa\}$
    \item $X_2 = \{b, ab, baa, abaa, aaaa\}$
    \item $X_3 = \{aa, ab, aab, bba\}$
    \item $X_4 = \{a, ba, bba, baab\}$
    \end{itemize}
  \end{question}


  
  \begin{question}
    Soit $u$ un mot de $\Sigma^*$, montrer que l'ensemble $\{u\}$ est un
    code si et seulement si $u \not= \epsilon$.
  \end{question}
  

  \begin{question}
    Soient $u$ et $v$ deux mots distincts de $\Sigma^*$, montrer que la
    partie $\{u,v\}$ est un code si et seulement si $u$ et $v$ ne
    commutent pas.
  \end{question}



  \begin{question}
    Soit $X$ une partie de $\Sigma^*$ ne contenant pas $\epsilon$ et telle
    qu'aucun mot de $X$ ne soit pr\'efixe propre d'un autre mot de
    $X$.  Montrer que $X$ est un code (un tel code est appel\'e code
    pr\'efixe).
  \end{question}

\end{exer}



\begin{exer}[Mots multiplicativement d�pendants]


  Deux mots $u$ et $v$ sont dits {\bf multiplicativement d�pendants}
  s'ils sont puissances d'un m�me troisi�me, c'est � dire s'il existe
  un mot $w$ et deux entiers $m$ et $n$ tels que
$$u = w^n \ \mbox{et} \ v = w^m$$


Deux mots $u$ et $v$ sont dits {\bf commutatifs} si $u v = v u$.


\begin{question}
Donner un exemple de deux mots commutatifs de longueur sup�rieure � 2
\end{question}

\begin{question}
Prouver la proposition suivante :

\begin{Prop}
Deux mots $u$ et $v$ commutent si et seulement si ils sont multiplicativement d�pendants.
\end{Prop}
\end{question}

\end{exer}

\end{document}


