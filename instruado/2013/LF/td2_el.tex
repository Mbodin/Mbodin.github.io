\documentclass[11pt,a4paper]{article}
  \usepackage[T1]{fontenc}%
  \usepackage[obeyspaces]{url} % \urlstyle{sf}%
  \usepackage{mathpazo}%
\usepackage[dvips]{epsfig}
\usepackage[latin1]{inputenc}

\usepackage{fancyhdr}
\usepackage[latin1]{inputenc}
\usepackage{graphicx,multicol,amsmath,amssymb,ifthen}
\usepackage[francais]{babel} \AutoSpaceBeforeFDP %
\usepackage{color}
\usepackage{framed}
\usepackage{stmaryrd}


\setlength{\oddsidemargin}{0cm}
\addtolength{\textwidth}{70pt}
\setlength{\topmargin}{0cm}
\addtolength{\textheight}{3cm}


\newcommand{\Correction}[1]
{
  \ifthenelse{\boolean{corrige}}
  {\definecolor{shadecolor}{gray}{0.80} \begin{shaded} {\small #1} \end{shaded}}{}
}



\newcommand{\titreTD}[3]{%
\title{Module #1 \\
#2 : #3
\Correction{Correction}}
\author{}
\date{}

\pagestyle{fancy}
\lhead{\sf MIT 1}
\rhead{\sf Module #1 \ #2 \ifthenelse{\boolean{corrige}}{, correction}{}}}



\newcounter{probleme} 
\newcounter{question}
\setcounter{probleme}{1} 
\setcounter{question}{1} 
\def\theprobleme{\Roman{probleme}}
\def\thequestion{\arabic{question}}

\newenvironment{probleme}[1]%
{\setcounter{question}{1}
\stepcounter{probleme}
\subsection*{\theprobleme. #1}}%
{}


\newcounter{excounter}
\newcounter{questcounter}[excounter]
\newenvironment{exer}[1][]%
{%
  \setcounter{question}{1}%
  \stepcounter{excounter}%
  \vspace*{0.6cm}\noindent { \texttt{{\bf Exercice \arabic{excounter}} #1}}\\
}{%
  \vspace{.3cm}%
}



\newenvironment{question}[1][]{\setcounter{numsubquestion}{1}%
\vspace*{0.5cm}\noindent{\bf {\large\theexcounter.\thequestion}. #1}}%
{\stepcounter{question}\medskip}

\newcounter{numdef}
\setcounter{numdef}{0}




\newenvironment{Def}[2]{%
\definecolor{shadecolor}{gray}{0.90}
\begin{shaded}\refstepcounter{numdef}%\vspace*{0.1cm}
    \noindent{ {\label{#2}\bf D�finition}~: {\it #1}} \\}%
{\end{shaded}\medskip}


\newenvironment{Th}[1]{%
\begin{framed}\vspace*{0.1cm}\noindent{ {\bf Th�or�me (#1)} :} \\}%
{\end{framed}}

\newenvironment{Lemma}[1]{%
\begin{framed}%\vspace*{0.1cm}
    \noindent{ {\bf Lemme (#1)} :} \\}%
{\end{framed}}


\newenvironment{Prop}[0]{%
\begin{framed}\vspace*{0.1cm}\noindent{ {\bf Proposition} :} \\}%
{\end{framed}}



\newcounter{numsubquestion}
\setcounter{numsubquestion}{1}
\newenvironment{subquestion}{\setcounter{numsubsubquestion}{1}%
\vspace*{0.2cm}\noindent{ {\bf\theexcounter.\thequestion . \thenumsubquestion}}.}%
{\stepcounter{numsubquestion}\medskip}


\newcounter{numsubsubquestion}
\newenvironment{subsubquestion}{\vspace*{0.1cm}\noindent{ {\bf\theexcounter.\thequestion.\thenumsubquestion.\thenumsubsubquestion}}.}%
{\stepcounter{numsubsubquestion}\medskip}


\let\bfseriesaux=\bfseries%
\renewcommand{\bfseries}{\sffamily \bfseriesaux}

\def\|{\leavevmode\url|}



\newenvironment{remarque}%
{\subparagraph{Remarque.}}%
{}

\newenvironment{programme}%
{\verbatim}{\endverbatim}

\newboolean{corrige} 

\setlength{\parindent}{0cm}

\newsavebox{\majbox}
\newlength{\majhaut}
\newenvironment{solution}{\color[gray]{0.5}%
  \begin{lrbox}{\majbox}
    \begin{minipage}[b]{.9\textwidth}}{%
   \end{minipage}
  \end{lrbox}
  \settoheight{\majhaut}{\usebox{\majbox}}
   \addtolength{\majhaut}{0.3mm}
  \ifthenelse{\boolean{corrige}}{%
    \rule{1mm}{\majhaut}\hspace*{2mm}\usebox{\majbox}}{}}

\newcommand{\st}{\ |\ }

\renewcommand\geq\geqslant
\renewcommand\leq\leqslant

 
  
\titreTD{Langages Formels}{TD 2}{R�siduels et automates finis} 

\setboolean{corrige}{true}


\newcommand\abs[1]{\left|#1\right|}
 

\parskip=-0.1\baselineskip \sloppy


\newcommand{\N}{\mathbb{N}}%
\newcommand{\D}{\mathbb{D}}%
\newcommand{\dist}{\mathop{\mathrm{d}}}%
\newcommand{\fix}{\mathop{\mathrm{Fix}}}%

\begin{document}

\thispagestyle{empty}
\maketitle

\begin{exer}[R\'esiduels]


  \begin{question}
    Calculer le r\'esiduel de $L$ par rapport � tout mot $u$ sur $\Sigma
    = \{a,b\}$ dans les exemples suivants~:
    \[ L = a^*b^* \hspace{2cm} L' = \{a^nb^n\st \ n \geq 0\}\]
    \vspace{-4ex}
  \end{question}

 
  \begin{question}
    Si $x$ est une lettre de $\Sigma$, que valent $x^{-1}(L_1 \cup L_2)$,
    $x^{-1}(L_1 L_2)$ et $x^{-1}L_1^*$ o� $L_1$ et $L_2$ sont deux langages sur $\Sigma$?
  \end{question}


\end{exer}


\begin{exer}[Codes et quotients]

On �tend la d�finition des r�siduels � gauche � des langages sur
$\Sigma^*$ : le {\bf quotient � gauche} d'un langage $L_1$ par un
langage $L_2$ est d�fini par $L_2^{-1}L_1 = \bigcup_{u \in L_2} u^{-1}
L_1$.

Soit $X$ un sous-ensemble de $\Sigma^+$. On d�finit la suite $(U_i)$ de
langages
\[
\left\{
\begin{array}{rcl}
U_1 &=& X^{-1}X \smallsetminus \{\varepsilon\}\\
U_{n+1} &=& X^{-1}U_{n} \cup U_{n}^{-1}X \mbox{ pour } n \geq 1\\
\end{array}
\right.
\]

\begin{question}
Montrer que pour tout $n \geq 1$, et pour tout $k \in [1,n]$, 
\[ \varepsilon \in U_n \iff \exists u \in U_k, \exists i,j \geq 0
\mbox{ tels que } uX^i \cap X^j \neq \emptyset \mbox{ avec } i+j+k=n
\]
\end{question}


  On appelle {\bf code} sur un alphabet $\Sigma$ tout langage $X$ sur
  $\Sigma$ tel que pour toutes familles $(x_i) \in X^{\llbracket 1, p \rrbracket}$
  et $(y_i) \in X^{\llbracket 1, q \rrbracket}$,
  $x_1x_2 \ldots x_p = y_1y_2 \ldots y_q$ entraine $p = q$ et
  $x_i = y_i$ pour tout $i$. Dire que $X$ est un code revient donc �
  dire que tout \'el\'ement de $X^*$ se factorise de mani�re unique sur
  $X$.

\begin{question}
En d�duire que $X$ est un code si et seulement si aucun des $U_i$ ne
contient $\varepsilon$.
\end{question}

\end{exer}



\begin{exer}
  �crire un automate d�terministe qui reconna�t les entiers �crits en base $2$ qui sont congrus � $1$
  modulo $3$.
\end{exer}


\begin{exer}[La m�thode de Thompson]
  On d�cide d'ajouter aux automates non d�terministes la possibilit� d'utiliser des
  $\varepsilon$-transitions. $\varepsilon$ est une �tiquette de transition qui correspond au mot vide.
  
  Par exemple, $b \in \mathcal{L}(A)$, avec $A$ l'automate ci-dessous.

  \begin{center}
    \input{autoepsilon.pstex_t}
  \end{center}

  \begin{question}
    Proposer un algorithme de d�terminisation des automates finis � $\varepsilon$-transitions et
    l'appliquer sur l'exemple ci-dessus.
  \end{question}


  \begin{question}
    Montrer que tout automate fini non d�terministe est �quivalent � un automate fini non d�terministe
    ayant un unique �tat initial et un unique �tat final.
  \end{question}


  \begin{question}
    Soient $A$ et $B$ deux automates finis. Construire des automates reconnaissant
\[\mathcal{L}(A) . \mathcal{L}(B)\hspace{2cm}\mathcal{L}(A) \cup \mathcal{L}(B)\hspace{2cm}\mathcal{L}(A)^*\]
  \end{question}


\end{exer}


\begin{exer}[D�terminisation]
  \begin{question}
    D�terminiser l'automate suivant :
    
    \begin{center}
      \input{auto2.pstex_t}
    \end{center}
  \end{question}

\begin{question}
  Nous allons maintenant calculer la complexit� au pire de la d�terminisation d'un automate en fonction de son
  nombre d'�tats. On a vu en cours que le d�terminis� d'un automate �
  $Q$ �tats a au plus  $2^{|Q|}$ �tats, nous allons d�tailler un exemple.
    Consid�rons l'automate $A$ suivant, qui reconna�t l'ensemble des mots de $\{a,b\}^*$ dont la
    $n^e$ lettre en partant de la fin est un $a$ (on suppose $n>0$):

    \begin{center}
      \input{auto3.pstex_t}
    \end{center}

    Soit $B = (Q, \Sigma = \{a,b\}, q_0, F, \delta)$ un automate fini d�terministe reconnaissant le m�me
    langage que $A$.

    \begin{subquestion}
      Montrer que $B$ est complet (i.e. si $u \in \Sigma^*$, alors $\delta(q_0,u)$ existe).
    \end{subquestion}


    \begin{subquestion}
      Prouver que la fonction $\varphi : \begin{array}[t]{l} \Sigma^{n-1}  \rightarrow  Q \\
	u \mapsto q_u = \delta(q_0,u) \end{array} $ est injective. 

En conclure que la
      d�terminisation de $A$ est au pire exponentielle en nombre d'�tats.
    \end{subquestion}


\end{question}



\end{exer}


\end{document}


