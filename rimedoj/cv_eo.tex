\documentclass[12pt,a4paper]{article}

\usepackage{ifxetex}
\ifxetex
\usepackage{xunicode,fontspec,xltxtra}
\usepackage{polyglossia}
\setmainlanguage{esperanto}
\setotherlanguage{english}
\setotherlanguage{french}
\setotherlanguage{spanish}
\newcommand{\en}[1]{\foreignlanguage{english}{\textit{#1}}}
\newcommand{\fr}[1]{\foreignlanguage{french}{\textit{#1}}}
\newcommand{\es}[1]{\foreignlanguage{spanish}{\textit{#1}}}
\else
\usepackage[utf8x]{inputenc}
\usepackage[T1]{fontenc}
\usepackage[esperanto, francais, english]{babel}
\fi

\usepackage{geometry}
\usepackage{url}
\usepackage{oldgerm}

\usepackage[inline]{enumitem}

\makeatletter
% From http://tex.stackexchange.com/questions/51086/add-second-enumeration-item-on-the-same-line
\newcommand{\inlineitem}[1][]{%
\ifnum\enit@type=\tw@
    {\descriptionlabel{#1}}
  \hspace{\labelsep}%
\else
  \ifnum\enit@type=\z@
       \refstepcounter{\@listctr}\fi
    \qquad\@itemlabel\hspace{\labelsep}%
\fi}
\makeatother

\ifxetex
\setmainfont[Mapping=tex-text, Ligatures={Common}, Numbers={OldStyle}]{Linux Libertine O}
\fi

\geometry{
	includeheadfoot,
	margin = 2.54cm,
	top = 1.5cm,
	bottom = 1.5cm
}

\newcommand{\ie}{t.e.\,}

\newenvironment{datecvsection}[1]%
               {\subsection*{#1}%
                 \noindent \begin{tabular}{@{}p{\annee}p{\texte}@{}}}
               {\end{tabular}}

\newenvironment{cvsection}[1]%
               {\subsection*{#1}}
               {}

\newenvironment{itemcvsection}[1]%
               {\subsection*{#1}\begin{itemize}}
               {\end{itemize}}

\def\arraystretch{1.5}

\begin{document}

% Pas de numéro de page
\pagestyle{empty}

% \annee est la largeur de la première colonne, c'est à dire celle
% contenant l'année scolaire. Elle est ici définie comme étant la
% largeur du texte « janvier-février ». À adapter le cas échéant.

\newlength{\annee}
\settowidth{\annee}{9999--9999}

% \texte est la largeur de la deuxième colonne. Elle est définie comme
% étant la largeur de la page moins celle de la première colonne.
% 2\tabcolsep est la largeur de l'espacement entre les colonnes.

\newlength{\texte}
\setlength{\texte}{\textwidth} \addtolength{\texte}{-\annee} 
	\addtolength{\texte}{-2\tabcolsep}

\begin{center} \Huge \textsc{Martin~Bodin} \end{center}

\parbox[c]{.5\textwidth}
{
  \noindent
  14, rue de la libération,
  Les Énaudières \\
  50~220 Saint-Quentin-sur-le-Homme \\
  \textsc{Francio}
}
\parbox[c]{.55\textwidth}
{
\begin{flushright}
  Naskiĝis la 30an de decembro, 1989 \\
  \noindent Telefonnumero~: \mbox{+33 6 27 00 58 84} \\
  \url{martin.bodin@ens-lyon.org} \\
  \url{https://www.doc.ic.ac.uk/~mbodin}
\end{flushright}
}


\begin{datecvsection}{Laboro}

    2018–2019 & Postdoktorado ĉe la \en{Imperial College}, en \textsc{Londono} (\textsc{Unuiĝinta Reĝlando}) pri la teoriaj fundamentoj de la analizado de vervivaj programlingvaĵoj. \\

    2017–2018 & Postdoktorado ĉe la \textsc{CMM} (la \en{Center of Mathematical Modeling}, parto de la \es{Universidad de Chile}) en \textsc{Santiago} (\textsc{Ĉilio}) pri formalizado de la programlingvaĵo \textsc{R} en la pruvilo \textsc{Coq}. \\

\end{datecvsection}

\begin{datecvsection}{Studaro}

    2012–2016 & Doktoriĝo informatika ĉe la \textsc{Irisa} en \textsc{Rennes} (\textsc{Francio}) estrigita de \textsc{Alan~Schmitt} kaj \textsc{Thomas~Jensen} pri korektpruvigitaj \textsc{Javaskript}aj analizoj. \\

	2010–2012 & \fr{Master d’Informatique} (\ie informatika Majstro), en la \fr{École Normale Supérieure de \textsc{Lyon}} (franca \fr{Grande École} — \ie ĉeflernejo de altnivela instruado — specialigita en la edukado de esploradistoj). \\

    2009–2010 & Licence d’Informatique, mention très bien (\ie komputika licencio, kun granda laŭdo) à l’École Normale Supérieure de \textsc{Lyon}. \\

    2007–2009 & Classes préparatoires aux Grandes Écoles au lycée \textsc{Henri~iv} en \textsc{Parizo} (\ie dujara intenskurso preparanta al la enirkonkurso de la \fr{grandes Écoles}). \\

    2007 & Baccalauréat spécialité Mathématiques, mention très bien (\ie bakalaŭrea ekzameno kies ĉefaj fakoj estas matematiko, fiziko kaj biologio, kun granda laŭdo). \\

\end{datecvsection}

\begin{cvsection}{Lingvoj}
\parbox{.4\textwidth}{
\begin{itemize}
   \item Denaska nivelo de la franca,
   \item Flua en Esperanto,
   \item Kompreneto de la germana.
\end{itemize}}
\parbox{.55\textwidth}{
\begin{itemize}
   \item Altnivela nivelo en la angla,
   \item Baznivelo en la portugala kaj la hispana,
\end{itemize}}
\end{cvsection}

\begin{datecvsection}{Laboraj spertoj}

	2013–2016 & Ano de la \textsc{SecCloud} projekto:  \url{http://www.seccloud.cominlabs.ueb.eu/} \\

	2012–2016 & Ano de la \textsc{JsCert} projekto:  \url{http://jscert.org/} \\

	2012 & Lernoservo ĉe \textsc{Inria} (\textsc{Rennes}) kun \textsc{Alan~Schmitt} kaj \textsc{Thomas~Jensen}:
	\en{\textit{A Certified \textsc{JavaScript} Interpreter}} \\

	2011 & Lernoservo ĉe \textsc{Vrije Universiteit} (\textsc{Amsterdam}) kun \textsc{Dimitri~Hendriks} kaj \textsc{Jörg~Endrullis}:
	\en{\textit{Proving Stream Equalities in \textsc{Coq}}} \\

	2011 & Lernoservo ĉe \textsc{pps} (\textsc{Paris}) kun \textsc{Stéphane Gimenez} kaj \textsc{Christine Tasson}:
	\en{\textit{Sequentialization of Proof Nets to Structures}} \\

	2010 & Partoprenado al la \textsc{Cartomancer} projekto:  \url{http://sourceforge.net/projects/cartomancer/} \\

	2010 & Lernoservo ĉe \textsc{Vérimag} (\textsc{Grenoble}) kun \textsc{David~Monniaux}:
	\textit{Détection de modes de fonctionnements d’un programme \textsc{Lustre}} \\

\end{datecvsection}

\begin{datecvsection}{Kongresoj}

  2019 & Martin \textsc{Bodin}, Philippa \textsc{Gardner}, Thomas \textsc{Jensen}, kaj Alan \textsc{Schmitt}. \textit{Skeletal Semantics kaj Their Interpretations}. Je POPL. \\

  2018 & Martin \textsc{Bodin}, Tomás \textsc{Diaz}, kaj Éric \textsc{Tanter}. \textit{A Trustworthy Mechanized Formalization of R}. Je DLS. \\

  2015 & Martin \textsc{Bodin}, Thomas \textsc{Jensen}, kaj Alan \textsc{Schmitt}. \textit{Certified Abstract Interpretation with Pretty-Big-Step Semantics}. Je CPP. \\

  2014 & Martin \textsc{Bodin}, Arthur \textsc{Charguéraud}, Daniele \textsc{Filaretti}, Philippa \textsc{Gardner}, Sergio \textsc{Maffeis}, Daiva \textsc{Naudžiūnienė}, Alan \textsc{Schmitt}, kaj Gareth \textsc{Smith}. \textit{A Trusted Mechanised JavaScript Specification}. Je POPL. \\

  2013 & Jörg \textsc{Endrullis}, Dimitri \textsc{Hendriks}, kaj Martin \textsc{Bodin}. \textit{Circular Coinduction in Coq Using Bisimulation-up-to Techniques}. Je ITP. \\

  2011 & David \textsc{Monniaux} kaj Martin \textsc{Bodin}. \textit{Modular Abstractions of Reactive Nodes using Disjunctive Invariants}. Je APLAS. \\

\end{datecvsection}

\begin{datecvsection}{Babilkongresoj}

  2018 & Martin \textsc{Bodin}. \textit{A Coq Formalisation of a Core of R}. Je CoqPL. \\

  2016 & Martin \textsc{Bodin}, Thomas \textsc{Jensen}, kaj Alan \textsc{Schmitt}. \textit{An Abstract Separation Logic for Interlinked Extensible Records}. Je JFLA. \\

  2014 & Martin \textsc{Bodin}, Thomas \textsc{Jensen}, kaj Alan \textsc{Schmitt}. \textit{Certified Abstract Interpretation with Pretty-Big-Step Semantics}. Je JFLA. \\

  2013 & Martin \textsc{Bodin}, Thomas \textsc{Jensen}, kaj Alan \textsc{Schmitt}. \textit{Pretty-big-step-semantics-based Certified Abstract Interpretation (Preliminary version)}. Je \textit{Semantics, Abstract Interpretation, kaj Reasoning about Programs: Essays Dedicated to David A. \textsc{Schmidt}} \\ % on the Occasion of his Sixtieth Birthday}, volume 129 of EPTCS, pages 360–383. \\

  2013 & Martin \textsc{Bodin} kaj Alan \textsc{Schmitt}. \textit{A Certified JavaScript Interpreter}. Je JFLA. \\

\end{datecvsection}

\begin{datecvsection}{Aliaj eldonaĵoj}

    2016 & Martin \textsc{Bodin}. \textit{Sémantique et analyse certifiées de JavaScript}. Doktoriĝdokumento. \\

\end{datecvsection}

\begin{itemcvsection}{Komputilaj kapabloj}

  \item Bona nivelo en la jenaj programlingvoj:  Coq, Javaskripto, R, \LaTeX, OCaml, ktp.
  \item Kutimuzanto de la sistemo GNU/Linux.
  \item Kapabla uzi Vim, Git, The~Gimp, Inkscape, ktp.

\end{itemcvsection}

\begin{datecvsection}{Asocia vivo}

    2017 & Esperantaj renkontiĝoj en \textsc{Santiago}. \\
    2014–2015 & Aktiva ano de \textsc{Espéranto-Rennes}. \\
	2012 & Kasisto de la teatra asocio \textsc{ASCREB} en \textsc{Rennes}. \\
	2010 & Sekretario de la asocio \textsc{INFO-ENS} en \textsc{Lyon}.

\end{datecvsection}

\begin{datecvsection}{Instruado}

    2018 & Models of Computation. Ekzercinstruado de L2a nivelo ĉe la \en{Imperial College London}. \\

    2017 & Introducción a Coq: Lógica, Tipos y Verificación. Ekzercinstruado de M2a nivelo ĉe la \es{Universidad de Chile}. \\

    2014 & Program-konceptado kaj pruvo. Ekzercinstruado de M1a nivelo ĉe la \fr{École Normale Supérieure} \\

    2013–2014 & Formalaj lingvaĵoj. Ekzercinstruado de L3a nivelo ĉe la \fr{École Normale Supérieure} \\

    2013 & Enkonduko al Scheme. Ekzercinstruado de L1a nivelo ĉe \textsc{Insa} de \textsc{Rennes} \\

	2012, 2014 & Superkontrolado de la algoritma malteoria ekzameno eniganta la \fr{École Normale Supérieure}.

\end{datecvsection}

\begin{itemcvsection}{Personaj interesigaĵoj}

  \item Ludoj: tabelaj, kartaj, rolaj, vesperenketoj, ktp.
  \item Esperantaj agadoj.

\end{itemcvsection}

\end{document}

