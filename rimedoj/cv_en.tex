\documentclass[12pt,a4paper]{article}

\usepackage{ifxetex}
\ifxetex
\usepackage{xunicode,fontspec,xltxtra}
\usepackage{polyglossia}
\setmainlanguage{english}
\setotherlanguage{french}
\setotherlanguage{spanish}
\newcommand{\fr}[1]{\foreignlanguage{french}{\textit{#1}}}
\newcommand{\es}[1]{\foreignlanguage{spanish}{\textit{#1}}}
\else
\usepackage[utf8x]{inputenc}
\usepackage[T1]{fontenc}
\usepackage[francais, english]{babel}
\newcommand{\fr}[1]{\foreignlanguage{francais}{\textit{#1}}}
\fi

\usepackage{geometry}
\usepackage{url}
\usepackage{oldgerm}

\usepackage[inline]{enumitem}

\makeatletter
% From http://tex.stackexchange.com/questions/51086/add-second-enumeration-item-on-the-same-line
\newcommand{\inlineitem}[1][]{%
\ifnum\enit@type=\tw@
    {\descriptionlabel{#1}}
  \hspace{\labelsep}%
\else
  \ifnum\enit@type=\z@
       \refstepcounter{\@listctr}\fi
    \qquad\@itemlabel\hspace{\labelsep}%
\fi}
\makeatother

\ifxetex
\setmainfont[Mapping=tex-text, Ligatures={Common}, Numbers={OldStyle}]{Linux Libertine O}
\fi

\geometry{
	includeheadfoot,
	margin = 2.54cm,
	top = 1.5cm,
	bottom = 1.5cm
}

\newcommand{\ie}{i.e.\,}

\newenvironment{datecvsection}[1]%
               {\subsection*{#1}%
                 \noindent \begin{tabular}{@{}p{\annee}p{\texte}@{}}}
               {\end{tabular}}

\newenvironment{cvsection}[1]%
               {\subsection*{#1}}
               {}

\newenvironment{itemcvsection}[1]%
               {\subsection*{#1}\begin{itemize}}
               {\end{itemize}}

\def\arraystretch{1.5}

\begin{document}

% Pas de numéro de page
\pagestyle{empty}

% \annee est la largeur de la première colonne, c'est à dire celle
% contenant l'année scolaire. Elle est ici définie comme étant la
% largeur du texte « janvier-février ». À adapter le cas échéant.

\newlength{\annee}
\settowidth{\annee}{9999—9999} % January--February}

% \texte est la largeur de la deuxième colonne. Elle est définie comme
% étant la largeur de la page moins celle de la première colonne.
% 2\tabcolsep est la largeur de l'espacement entre les colonnes.

\newlength{\texte}
\setlength{\texte}{\textwidth} \addtolength{\texte}{-\annee} 
	\addtolength{\texte}{-2\tabcolsep}

\begin{center} \Huge \textsc{Martin~Bodin} \end{center}

\parbox{0.5\textwidth}
{
  \noindent
  14, rue de la libération,
  Les Énaudières \\
  50~220 Saint-Quentin-sur-le-Homme \\
  \textsc{France}
}
\parbox{.55\textwidth}
{
\begin{flushright}
  Born the 30\(^{th}\) of December 1989. \\
  \noindent Phone Number: \mbox{+33 6 27 00 58 84} \\
  \url{martin.bodin@ens-lyon.org} \\
  \url{https://www.doc.ic.ac.uk/~mbodin}
\end{flushright}
}


\begin{datecvsection}{Employment}

    2018–2019 & Postdoc at the \textsc{Imperial College}, in \textsc{London} (\textsc{United Kingdom}) on the theoretical foundations of the analysis of real-world languages. \\

    2017–2018 & Postdoc at the \textsc{CMM} (the Center of Mathematical Modeling, part of the \es{Universidad de Chile}), in \textsc{Santiago} (\textsc{Chile}) on the formalisation of the R programming language in the Coq proof assistant. \\

\end{datecvsection}

\begin{datecvsection}{Related Experience}

	2013–2016 & Member of the \textsc{SecCloud} project:  \url{www.seccloud.cominlabs.ueb.eu/}. \\

	2012–2016 & Member of the \textsc{JsCert} project:  \url{jscert.org/}. \\

	2012 & Internship in \textsc{Inria} at \textsc{Rennes} with \textsc{Alan~Schmitt} and \textsc{Thomas~Jensen}:
	\textit{A Certified JavaScript Interpreter}. \\

	2011 & Internship in \textsc{Vrije~Universiteit} at \textsc{Amsterdam} with \textsc{Dimitri~Hendriks} and \textsc{Jörg~Endrullis}:
	\textit{Proving Stream Equalities in Coq}. \\

	2011 & Internship in \fr{\sc PPS} at \textsc{Paris} with \textsc{Stéphane~Gimenez} and \textsc{Christine~Tasson}:
	\textit{Sequentialization of proof nets to structures}. \\

	2010 & Internship in \fr{\sc Vérimag} at \textsc{Grenoble} with \textsc{David~Monniaux}:
        \fr{Détection de modes de fonctionnements d’un programme \textsc{Lustre}}. \\

\end{datecvsection}

\begin{datecvsection}{Education}

    2012–2016 & Ph.D. in Computer Sciences at \textsc{IRISA}, in \textsc{Rennes} (\textsc{France}), under the direction of \textsc{Alan~Schmitt} and \textsc{Thomas~Jensen} on certified analyses of JavaScript. \\

	2010–2012 & \fr{Master d’Informatique} (\ie a Computer Science
	Master) in the \fr{École Normale Supérieure de Lyon} (a French
	\fr{grande École}—\ie a leading institution of higher
    education—specialized in the formation of searchers). \\

	2009–2010 & \fr{Licence d’Informatique, mention très bien} (\ie
	a Computer Science Licence degree, magna cum laude) in the
	\fr{École Normale Supérieure de Lyon}. \\

	2007–2009 & \fr{Classes préparatoires aux grandes Écoles} in the
	\fr{lycée Henri~IV} in Paris (a two-years intensive course
	preparing to the competitive entrance of \fr{grandes Écoles}). \\

	2007 & \fr{Baccalauréat Scientifique, mention très bien}
	(equivalent to British A-levels with a major in Mathematics, Physics
	and Biology, magna cum laude). \\

\end{datecvsection}

\begin{cvsection}{Languages}
\parbox{.4\textwidth}{
\begin{itemize}
   \item Native in French;
   \item Fluent in Esperanto;
   \item Notions of German.
\end{itemize}}
\parbox{.5\textwidth}{
\begin{itemize}
   \item Good level of English;
   \item Basic level of Portuguese and Spanish;
\end{itemize}
~\\}
\end{cvsection}

\begin{datecvsection}{Conferences}

  2019 & Martin \textsc{Bodin}, Philippa \textsc{Gardner}, Thomas \textsc{Jensen}, and Alan \textsc{Schmitt}. \textit{Skeletal Semantics and Their Interpretations}. In POPL. \\

  2018 & Martin \textsc{Bodin}, Tomás \textsc{Diaz}, and Éric \textsc{Tanter}. \textit{A Trustworthy Mechanized Formalization of R}. In DLS. \\

  2015 & Martin \textsc{Bodin}, Thomas \textsc{Jensen}, and Alan \textsc{Schmitt}. \textit{Certified Abstract Interpretation with Pretty-Big-Step Semantics}. In CPP. \\

  2014 & Martin \textsc{Bodin}, Arthur \textsc{Charguéraud}, Daniele \textsc{Filaretti}, Philippa \textsc{Gardner}, Sergio \textsc{Maffeis}, Daiva \textsc{Naudžiūnienė}, Alan \textsc{Schmitt}, and Gareth \textsc{Smith}. \textit{A Trusted Mechanised JavaScript Specification}. In POPL. \\

  2013 & Jörg \textsc{Endrullis}, Dimitri \textsc{Hendriks}, and Martin \textsc{Bodin}. \textit{Circular Coinduction in Coq Using Bisimulation-up-to Techniques}. In ITP. \\

  2011 & David \textsc{Monniaux} and Martin \textsc{Bodin}. \textit{Modular Abstractions of Reactive Nodes using Disjunctive Invariants}. In APLAS. \\

\end{datecvsection}

\begin{datecvsection}{Workshops}

  2018 & Martin \textsc{Bodin}. \textit{A Coq Formalisation of a Core of R}. In CoqPL. \\

  2016 & Martin \textsc{Bodin}, Thomas \textsc{Jensen}, and Alan \textsc{Schmitt}. \textit{An Abstract Separation Logic for Interlinked Extensible Records}. In JFLA. \\

  2014 & Martin \textsc{Bodin}, Thomas \textsc{Jensen}, and Alan \textsc{Schmitt}. \textit{Certified Abstract Interpretation with Pretty-Big-Step Semantics}. In JFLA. \\

  2013 & Martin \textsc{Bodin}, Thomas \textsc{Jensen}, and Alan \textsc{Schmitt}. \textit{Pretty-big-step-semantics-based Certified Abstract Interpretation (Preliminary version)}. In \textit{Semantics, Abstract Interpretation, and Reasoning about Programs: Essays Dedicated to David A. \textsc{Schmidt}} \\ % on the Occasion of his Sixtieth Birthday}, volume 129 of EPTCS, pages 360–383. \\

  2013 & Martin \textsc{Bodin} and Alan \textsc{Schmitt}. \textit{A Certified JavaScript Interpreter}. In JFLA. \\

\end{datecvsection}

\begin{datecvsection}{Other Publications}

    2016 & Martin \textsc{Bodin}. \textit{Certified semantics and analysis of JavaScript}. Ph.D. thesis. \\

\end{datecvsection}

\begin{itemcvsection}{Computer Proficiencies}

	\item Good commands in the following programming languages:  Coq, JavaScript, R, \LaTeX, OCaml, etc.
	\item Familiar with GNU/Linux as an operating system.
	\item Able to use Vim, Git, The~Gimp, Inkscape, etc.

\end{itemcvsection}

\begin{datecvsection}{Association Experience}

    2017 & Esperanto Meetups in \textsc{Santiago}. \\
    2014–2015 & Active member of \textsc{Espéranto-Rennes}. \\
	2012 & Treasurer of the theater association \textsc{ASCREB} in \textsc{Rennes}. \\
	2010 & Secretary of the \textsc{INFO-ENS} association in \textsc{Lyon}. \\

\end{datecvsection}

\begin{datecvsection}{Teaching}

    2018 & Models of Computation. L2 tutoring at the Imperial College London. \\

    2017 & Introduction to Coq: Logic, Types, and Verification. M2 tutoring at the \es{Universidad de Chile}. \\

    2014 & Software Design and Verification. M1 tutoring at the \fr{École Normale Supérieure}. \\

    2013–2014 & Formal Languages. L3 tutoring at the \fr{École Normale Supérieure}. \\

    2013 & Introduction to Scheme. L1 tutoring at the \textsc{Insa} of \textsc{Rennes}. \\

	2012, 2014 & Monitoring the algorithmic practice examination for the entrance of the \fr{École Normale Supérieure}. \\

\end{datecvsection}

\begin{itemcvsection}{Personnal Interests}

	\item Board games, card games, improvisation games, role playing games, etc.
    \item Esperanto related activities.

\end{itemcvsection}

\end{document}

