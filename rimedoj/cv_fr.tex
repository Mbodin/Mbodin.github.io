\documentclass[12pt,a4paper]{article}

\usepackage{ifxetex}
\ifxetex
\usepackage{xunicode,fontspec,xltxtra}
\else
\usepackage[utf8x]{inputenc}
\usepackage[T1]{fontenc}
\fi

\usepackage[francais, english]{babel}
\usepackage{geometry}
\usepackage{url}
\usepackage{oldgerm}

\usepackage[inline]{enumitem}

\makeatletter
% From http://tex.stackexchange.com/questions/51086/add-second-enumeration-item-on-the-same-line
\newcommand{\inlineitem}[1][]{%
\ifnum\enit@type=\tw@
    {\descriptionlabel{#1}}
  \hspace{\labelsep}%
\else
  \ifnum\enit@type=\z@
       \refstepcounter{\@listctr}\fi
    \qquad\@itemlabel\hspace{\labelsep}%
\fi}
\makeatother

\ifxetex
\setmainfont[Mapping=tex-text, Ligatures={Common}, Numbers={OldStyle}]{Linux Libertine O}
\fi

\geometry{
	includeheadfoot,
	margin = 2.54cm,
	top = 1.5cm,
	bottom = 1.5cm
}

\newcommand{\en}[1]{\foreignlanguage{english}{\textit{#1}}}

\newcommand{\ie}{i.e.\,}

\newenvironment{datecvsection}[1]%
               {\subsection*{#1}%
                 \noindent \begin{tabular}{@{}p{\annee}p{\texte}@{}}}
               {\end{tabular}}

\newenvironment{cvsection}[1]%
               {\subsection*{#1}}
               {}

\newenvironment{itemcvsection}[1]%
               {\subsection*{#1}\begin{itemize}}
               {\end{itemize}}

\def\arraystretch{1.5}

\begin{document}

% Pas de numéro de page
\pagestyle{empty}

% \annee est la largeur de la première colonne, c'est à dire celle
% contenant l'année scolaire. Elle est ici définie comme étant la
% largeur du texte « janvier-février ». À adapter le cas échéant.

\newlength{\annee}
\settowidth{\annee}{9999--9999}

% \texte est la largeur de la deuxième colonne. Elle est définie comme
% étant la largeur de la page moins celle de la première colonne.
% 2\tabcolsep est la largeur de l'espacement entre les colonnes.

\newlength{\texte}
\setlength{\texte}{\textwidth} \addtolength{\texte}{-\annee} 
	\addtolength{\texte}{-2\tabcolsep}

\begin{center} \Huge \textsc{Martin~Bodin} \end{center}

\parbox[c]{.5\textwidth}
{
  \noindent
  Huerfanos 1400, Departamento 815B \\
  Santiago centro \\
  \textsc{Chili}
% Les Énaudières \\
% 14, rue de la Libération \\
% 50 220, \textsc{Saint-Quentin-sur-le-Homme} \\
% \textsc{France}
}
\parbox[c]{.55\textwidth}
{
\begin{flushright}
  Né le 30 Décembre 1989 \\
  \noindent Téléphone~: \mbox{06 27 00 58 84} \\
  % \url{martin.bodin@ens-lyon.org} \\
  \url{mbodin@dim.uchile.cl} \\
  \url{http://www.cmm.uchile.cl/~mbodin}
\end{flushright}
}


\begin{datecvsection}{Études}

    2017 & Postdoc au \textsc{CMM}, à \textsc{Santiago} (\textsc{Chili}), sur la formalisation du langage de programmation \textsc{R}. \\

    2012–2016 & Doctorat en Informatique à l’\textsc{IRISA}, à \textsc{Rennes}, sous la direction d’\textsc{Alan~Schmitt} et \textsc{Thomas~Jensen} sur les analyses certifiées du \textsc{JavaScript}. \\

	2010–2012 & Master d’Informatique, mention bien à l’École Normale Supérieure de \textsc{Lyon}. \\

	2009–2010 & Licence d’Informatique, mention très bien à l’École Normale Supérieure de \textsc{Lyon}. \\

	2007–2009 & Classes préparatoires aux Grandes Écoles au lycée \textsc{Henri~iv} à \textsc{Paris}. \\

	2007 & Baccalauréat spécialité Mathématiques, mention très bien. \\

\end{datecvsection}

\begin{itemcvsection}{Langues}

   \item Français langue maternelle,
   \item Bon niveau d’anglais,
   \item Très bon niveau d’espéranto,
   \item Bas niveau de portugais et d’espagnol,
   \item Notions d’allemand.

\end{itemcvsection}

\begin{datecvsection}{Expérience personnelle}

	2013–2016 & Membre du projet \textsc{SecCloud}~:  \url{http://www.seccloud.cominlabs.ueb.eu/} \\

	2012–2016 & Membre du projet \textsc{JsCert}~:  \url{http://jscert.org/} \\

	2012 & Stage à \textsc{Inria} (\textsc{Rennes}) avec \textsc{Alan~Schmitt} et \textsc{Thomas~Jensen}~:
	\en{\textit{A Certified \textsc{JavaScript} Interpreter}} \\

	2011 & Stage à \textsc{Vrije Universiteit} (\textsc{Amsterdam}) avec \textsc{Dimitri~Hendriks} et \textsc{Jörg~Endrullis}~:
	\en{\textit{Proving Stream Equalities in \textsc{Coq}}} \\

	2011 & Stage à \textsc{pps} (\textsc{Paris}) avec \textsc{Stéphane Gimenez} et \textsc{Christine Tasson}~:
	\en{\textit{Sequentialization of Proof Nets to Structures}} \\

	2010 & Participation au projet \textsc{Cartomancer}~:  \url{http://sourceforge.net/projects/cartomancer/} \\

	2010 & Stage à \textsc{Vérimag} (\textsc{Grenoble}) avec \textsc{David~Monniaux}~:
	\textit{Détection de modes de fonctionnements d’un programme \textsc{Lustre}} \\

\end{datecvsection}

\begin{datecvsection}{Conférences}

  2015 & \textit{Certified Abstract Interpretation with Pretty-Big-Step Semantics}, Martin \textsc{Bodin}, Thomas \textsc{Jensen} et Alan \textsc{Schmitt} (CPP~2015). \\

  2014 & \textit{A Trusted Mechanised JavaScript Specification}, Martin \textsc{Bodin}, Arthur \textsc{Charguéraud}, Daniele \textsc{Filaretti}, Philippa \textsc{Gardner}, Sergio \textsc{Maffeis}, Daiva \textsc{Naudziuniene}, Alan \textsc{Schmitt} et Gareth \textsc{Smith} (POPL~2014). \\

  2013 & \textit{Circular Coinduction in Coq Using Bisimulation-up-to Techniques}, Jörg \textsc{Endrullis}, Dimitri \textsc{Hendriks} et Martin \textsc{Bodin} (ITP~2013). \\

  2011 & \textit{Modular Abstractions of Reactive Nodes using Disjunctive Invariants}, David \textsc{Monniaux} and Martin \textsc{Bodin} (APLAS~2011). \\

\end{datecvsection}

\begin{datecvsection}{Ateliers}

  2016 & \textit{An Abstract Separation Logic for Interlinked Extensible Records}, Martin \textsc{Bodin}, Thomas \textsc{Jensen} et Alan \textsc{Schmitt} (JFLA~2016). \\

  2014 & \textit{Certified Abstract Interpretation with Pretty-Big-Step Semantics}, Martin \textsc{Bodin}, Thomas \textsc{Jensen} et Alan \textsc{Schmitt} (JFLA~2014). \\

  2013 & \textit{Pretty-big-step-semantics-based Certified Abstract Interpretation (Preliminary version)}, Martin \textsc{Bodin}, Thomas \textsc{Jensen} et Alan \textsc{Schmitt} in \textit{Semantics, Abstract Interpretation, and Reasoning about Programs: Essays Dedicated to David A. \textsc{Schmidt} on the Occasion of his Sixtieth Birthday}, volume 129 of EPTCS, pages 360–383. \\

  2013 & \textit{A Certified JavaScript Interpreter}, Martin \textsc{Bodin} and Alan \textsc{Schmitt} (JFLA~2013). \\

\end{datecvsection}

\begin{itemcvsection}{Compétences informatiques}

  \item Bon niveau dans les langages de programmation suivants~:  \textsc{Coq}, \textsc{JavaScript}, \LaTeX, \textsc{OCaml}\ldots
  \item Familier avec le système d’exploitation \textsc{GNU/Linux}.
  \item Capable d’utiliser \textsc{Vim}, \textsc{Git}, \textsc{The~Gimp}, \textsc{Inkscape}, \textsc{Graphviz}\ldots

\end{itemcvsection}

\begin{datecvsection}{Vie associative}

    2014–2015 & Membre actif d’\textsc{Espéranto-Rennes}. \\
	2012 & Trésorier de l’association de théâtre l’\textsc{ASCREB} à \textsc{Rennes}. \\
	2010 & Secrétaire de l’association \textsc{INFO-ENS} à \textsc{Lyon}. \\

\end{datecvsection}

\begin{datecvsection}{Expériences diverses}

    2014 & Conception et vérification formelle de programmes, travaux dirigés de M1 à l’École Normale Supérieure \\

    2013–2014 & Langages formels, travaux dirigés de L3 à l’École Normale Supérieure \\

    2013 & Introduction à \textsc{Scheme}, travaux pratiques et dirigés de L1 à l’\textsc{Insa} de \textsc{Rennes} \\

	2012, 2014 & Surveillance de l’épreuve pratique d’algorithmique et de programmation du concours d’entrée à l’École Normale Supérieure.

\end{datecvsection}

\begin{itemcvsection}{Intérêts personnels}

  \item Jeux de plateaux, de cartes, de rôles, d’improvisation, soirées enquêtes, etc.
  \item Activités espérantistes.

\end{itemcvsection}

\end{document}

