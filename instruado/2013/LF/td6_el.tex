\documentclass[11pt,a4paper]{article}
  \usepackage[T1]{fontenc}%
  \usepackage[obeyspaces]{url} % \urlstyle{sf}%
  \usepackage{mathpazo}%
\usepackage[dvips]{epsfig}
\usepackage[latin1]{inputenc}
\usepackage{textcomp}

\usepackage{fancyhdr}
\usepackage[latin1]{inputenc}
\usepackage{graphicx,multicol,amsmath,amssymb,ifthen}
\usepackage[francais]{babel} \AutoSpaceBeforeFDP %
\usepackage{color}
\usepackage{framed}
\usepackage{stmaryrd}


\setlength{\oddsidemargin}{0cm}
\addtolength{\textwidth}{70pt}
\setlength{\topmargin}{0cm}
\addtolength{\textheight}{3cm}


\newcommand{\Correction}[1]
{
  \ifthenelse{\boolean{corrige}}
  {\definecolor{shadecolor}{gray}{0.80} \begin{shaded} {\small #1} \end{shaded}}{}
}



\newcommand{\titreTD}[3]{%
\title{Module #1 \\
#2 : #3
\Correction{Correction}}
\author{}
\date{}

\pagestyle{fancy}
\lhead{\sf MIT 1}
\rhead{\sf Module #1 \ #2 \ifthenelse{\boolean{corrige}}{, correction}{}}}



\newcounter{probleme} 
\newcounter{question}
\setcounter{probleme}{1} 
\setcounter{question}{1} 
\def\theprobleme{\Roman{probleme}}
\def\thequestion{\arabic{question}}

\newenvironment{probleme}[1]%
{\setcounter{question}{1}
\stepcounter{probleme}
\subsection*{\theprobleme. #1}}%
{}


\newcounter{excounter}
\newcounter{questcounter}[excounter]
\newenvironment{exer}[1][]%
{%
  \setcounter{question}{1}%
  \stepcounter{excounter}%
  \vspace*{0.6cm}\noindent { \texttt{{\bf Exercice \arabic{excounter}} #1}}\\
}{%
  \vspace{.3cm}%
}



\newenvironment{question}[1][]{\setcounter{numsubquestion}{1}%
\vspace*{0.5cm}\noindent{\bf {\large\theexcounter.\thequestion}. #1}}%
{\stepcounter{question}\medskip}

\newcounter{numdef}
\setcounter{numdef}{0}




\newenvironment{Def}[2]{%
\definecolor{shadecolor}{gray}{0.90}
\begin{shaded}\refstepcounter{numdef}%\vspace*{0.1cm}
    \noindent{ {\label{#2}\bf D�finition}~: {\it #1}} \\}%
{\end{shaded}\medskip}


\newenvironment{Th}[1]{%
\begin{framed}\vspace*{0.1cm}\noindent{ {\bf Th�or�me (#1)} :} \\}%
{\end{framed}}

\newenvironment{Lemma}[1]{%
\begin{framed}%\vspace*{0.1cm}
    \noindent{ {\bf Lemme (#1)} :} \\}%
{\end{framed}}


\newenvironment{Prop}[0]{%
\begin{framed}\vspace*{0.1cm}\noindent{ {\bf Proposition} :} \\}%
{\end{framed}}



\newcounter{numsubquestion}
\setcounter{numsubquestion}{1}
\newenvironment{subquestion}{\setcounter{numsubsubquestion}{1}%
\vspace*{0.2cm}\noindent{ {\bf\theexcounter.\thequestion . \thenumsubquestion}}.}%
{\stepcounter{numsubquestion}\medskip}


\newcounter{numsubsubquestion}
\newenvironment{subsubquestion}{\vspace*{0.1cm}\noindent{ {\bf\theexcounter.\thequestion.\thenumsubquestion.\thenumsubsubquestion}}.}%
{\stepcounter{numsubsubquestion}\medskip}


\let\bfseriesaux=\bfseries%
\renewcommand{\bfseries}{\sffamily \bfseriesaux}

\def\|{\leavevmode\url|}



\newenvironment{remarque}%
{\subparagraph{Remarque.}}%
{}

\newenvironment{programme}%
{\verbatim}{\endverbatim}

\newboolean{corrige} 

\setlength{\parindent}{0cm}

\newsavebox{\majbox}
\newlength{\majhaut}
\newenvironment{solution}{\color[gray]{0.5}%
  \begin{lrbox}{\majbox}
    \begin{minipage}[b]{.9\textwidth}}{%
   \end{minipage}
  \end{lrbox}
  \settoheight{\majhaut}{\usebox{\majbox}}
   \addtolength{\majhaut}{0.3mm}
  \ifthenelse{\boolean{corrige}}{%
    \rule{1mm}{\majhaut}\hspace*{2mm}\usebox{\majbox}}{}}

\newcommand{\st}{\ |\ }

\renewcommand\geq\geqslant
\renewcommand\leq\leqslant

 
  
\titreTD{Langages Formels}{TD 6}{Lemme de l'�toile et Lemme d'Ogden} 

\setboolean{corrige}{true}

% \documentclass[11pt]{article}

 

\parskip=0.3\baselineskip \sloppy


\newcommand{\fleche}{\rightarrow}%
\newcommand{\Rat}{\mathsf{Rat}}%
\newcommand{\N}{\mathbb{N}}%
\newcommand{\D}{\mathbb{D}}%
\newcommand{\dist}{\mathop{\mathrm{d}}}%
\newcommand{\fix}{\mathop{\mathrm{Fix}}}%
\newcommand{\vertbar}{\ \vert \ }%
%\newcommand{\vdashl}{\vdash\mspace{-6 mu}\textthreequartersemdash}%
\def\vdashl{\mathrel {\scriptstyle |} \joinrel\mathrel {\scriptstyle -}\joinrel\mathrel {\scriptstyle -}\joinrel\mathrel {\scriptstyle -}}
\def\vdashle{\mathrel {\scriptstyle |} \joinrel\mathrel {\scriptstyle \overset{*}{-}}\joinrel\mathrel {\scriptstyle -}\joinrel\mathrel {\scriptstyle -}}
\def\vdashlp{\mathrel {\scriptstyle |} \joinrel\mathrel {\scriptstyle \overset{+}{-}}\joinrel\mathrel {\scriptstyle -}\joinrel\mathrel {\scriptstyle -}}

\begin{document}

\thispagestyle{empty}
\maketitle


\begin{exer}[Point d'\emph{�toile noire}, le lemme de l'�toile nous suffit !]

Montrer que les langages suivants ne sont pas alg�briques.
\begin{enumerate}
\item $L_1 = \{a^ib^jc^k,\ i<j<k\}$ ;
\item $L_2 = \{a^nb^nc^m,\ n \leq  m \leq  2n\}$ ;
\item $L_3 = \{a^{2^n},\ n \geq 0 \}$ ;
\item $L_4 = \{a^{n^2},\ n \geq 0 \}$ ;
\end{enumerate}

\Correction{
}



\end{exer}


\begin{exer}[Lemme d'Ogden]

L'objectif de cet exercice est de montrer une version plus forte du lemme de
l'�toile pour les langages alg�briques~:
\begin{Lemma}{Ogden}
  Soit $L$ un langage alg�brique. Il existe un entier $N$ tel que pour
  tout mot $z\in L$ dans lequel on marque au moins $N$ positions
  distinctes, il est possible de d�composer $z$ sous la forme
  $z=uxvyw$ avec
  \begin{itemize}
  \item $x$ ou $y$ contient au moins une position marqu�e,
  \item $xvy$ contient au plus $N$ positions marqu�es,
  \item pour tout $i\geq 0$, $ux^ivy^iw\in L$.
  \end{itemize}
\end{Lemma}

\begin{question}
  On se donne une grammaire alg�brique propre $G$ engendrant un langage $L$.
  Montrer qu'il existe une grammaire sous forme normale de Chomsky (CNF)
  reconnaissant le langage $L-\{\varepsilon\}$.
\end{question}

$\blacktriangleright$~Rappel : Une grammaire CNF est une grammaire dont toutes les
productions sont de la forme
  \[A\rightarrow BC\quad \textrm{ou}\quad A\rightarrow a\]
  

\Correction{
}





\begin{Def}{}{}
  Soit $T$ un sous-arbre d'un arbre de d�rivation selon une grammaire
  CNF. On suppose marqu�es certaines feuilles de $T$. On appelle
  \emph{embranchement} un n\oe ud de $T$ ayant deux fils, tel que
  chacun de ses fils contienne au moins une feuille marqu�e.
\end{Def}

\begin{question}
  Soit $T$ un sous-arbre d'un arbre de d�rivation selon une grammaire
  CNF. On suppose qu'au moins $2^h$ feuilles distinctes de $T$ ont �t�
  marqu�es.
  
  Montrer qu'il existe un chemin, de la racine � une feuille, passant
  par au moins $h$ embranchements.
\end{question}

\Correction{
}
    






\begin{question}
  On consid�re une grammaire CNF $G$ engendrant le langage $L$.
  Montrer qu'il existe un entier $N$ tel que :
  \begin{itemize}
  \item pour tout mot $w\in L$ dans lequel on marque au moins $N$ positions,
  \item pour tout arbre de d�rivation de $w$,
  \end{itemize}
  il existe deux embranchements $b_1$ et $b_2$ tels que
  \begin{itemize}
  \item[$\bullet$] $b_1$ est un anc�tre de $b_2$,
  \item[$\bullet$] $b_1$ est un anc�tre d'au plus $N$ feuilles marqu�es,
  \item[$\bullet$] $b_1$ et $b_2$ sont �tiquet�s par la m�me variable.
  \end{itemize}
\end{question}


\Correction{
}


\begin{question}
  En d�duire le lemme d'Ogden.  
\end{question} 

\Correction{
}


\begin{question} 
  On s'int�resse au langage $L_5=\{a^ib^jc^kd^l\quad |\quad i=0\quad \textrm{ou}\quad 
  j=k=l\}$.


  \begin{subquestion}
    Montrer que pour tout $N\in\N^+$ et tout mot $z\in L_5$ avec $|z| \ge N$, il existe une
    d�composition $z=uxvyw$ telle que
    \begin{itemize}
    \item $|xy|\geq 1$
    \item $|xvy|\leq N$
    \item pour tout $i\geq 0$, $ux^ivy^iw\in L_5$.
    \end{itemize}
  \end{subquestion}
  
  \Correction{
  }



  \begin{subquestion}
    Montrer que $L_5$ n'est pas alg�brique.
  \end{subquestion}

  \Correction{
  }
\end{question}


\end{exer}


\begin{exer}[Couper la poire en deux n'est pas toujours rationnel.]

  Pour tout langage $L$ sur $\Sigma$, on d\'efinit
  \[\frac{1}{2}L=\left\{x\in\Sigma^* \middle| \exists y\in \Sigma^*,\ |x|=|y|\ \textrm{et}\ xy\in L\right\}\]

  \begin{question}
    Montrer que le langage $L_6=\left\{a^nb^nc^md^{3m} \middle| n,m\geq 1\right\}$
    est alg\'ebrique.
  \end{question}

  \Correction{
  }
  
  \begin{question}
    Calculer $\frac12L_6$.
  \end{question}

  \Correction{
  }




  \begin{question}
    Montrer que $\frac12L_6$ n'est pas alg\'ebrique.
  \end{question}

  \Correction{
}




\end{exer}





\end{document}


